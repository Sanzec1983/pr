\documentclass{article}
\usepackage[T2A]{fontenc}
\usepackage[utf8]{inputenc}
\usepackage[english,russian]{babel}
\usepackage{tikz}
\usepackage[european,cuteinductors,smartlabels]{circuitikz}
\usepackage{}
\title{Практическая работа №6}
\author{студент:    группы    }
% Конец преамбулы
\begin{document}
\maketitle

\begin{circuitikz}
%Рисование тр+тир
\draw (0,0) node [npn,bodydiode]{VT1};
\draw (0,-3.5) node [npn,bodydiode]{VT2};
%Рисование тр
\draw (2,0) node[npn,bodydiode]{VT3};
\draw (4,0) node[npn,bodydiode]{VT4};
%Рисование тр
\draw (2,-3.5) node [npn,bodydiode]{VT5};
\draw (4,-3.5) node [npn,bodydiode]{VT6};
%рисование вертикальных линий
%1 ветка
\draw (0,-0.3)--(0,-2.8);
\draw (0,0.3)--(0,2);
\draw (0,-4)--(0,-5);
\draw ((0,2.1) [-*];%точка
\draw ((0,-4.9) [-*];%точка
%2 ветка
\draw (2,-0.3)--(2,-2.8);
\draw (2,0.3)--(2,2);
\draw (2,-4)--(2,-5);
\draw ((2,2.1) [-*];%точка
\draw ((2,-4.9) [-*];%точка
%3 ветка
\draw (4,-0.3)--(4,-2.8);
\draw (4,0.3)--(4,2);
\draw (4,-4)--(4,-5);
%Конденсатор
\draw (-2,-3.5) to[C=C]  (-2,0);
%Рисование отводоы конденсатора
\draw (-2,0)--(-2,2);
\draw (-2,-2)--(-2,-5);
\draw (-2,2)--(4,2);
\draw (-2,-5)--(4,-5);
%Выводы к катушке
\draw (0,-0.7)--(5,-0.7);
\draw ((0,-0.6) [-*];%точка
\draw (2,-1.8)--(5,-1.8);
\draw (2,-1.7) [-*];%точка
\draw (4,-2.9)--(5,-2.9);
\draw ((4,-2.8) [-*];%точка
%рисование катушки
\draw (5,-0.7) to [L,l={L1}] (8,-0.7);  
\draw (5,-1.8) to [L,l={L2}] (8,-1.8);  
\draw (5,-2.9) to [L,l={L3}] (8,-2.9);  
%Ноль
\draw (8,-0.7)-- (8,-2.9);
\draw ((8,-1.7) [-*];%точка
\end{circuitikz}
\end{document}
