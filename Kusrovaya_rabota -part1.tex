\documentclass[russian,utf8,nocolumnxxxi,nocolumnxxxii]{eskdtext}
\usepackage[T1,T2A]{fontenc}
\usepackage[utf8]{inputenc}
\usepackage{amssymb,amsmath}
\usepackage{tikz}
\usepackage{siunitx}
\usepackage[american,cuteinductors,smartlabels]{circuitikz}
\usepackage[backend=biber]{biblatex}
\addbibresource{error_estimation_otchet.bib}
\usepackage[]{hyperref}
\hypersetup{
colorlinks=true,
}

\usepackage{textcomp}
\newcommand{\No}{\textnumero}
%Титульный лист
\ESKDdepartment{Федеральное агенство по образованию}

\ESKDcompany{Санкт-Петербургский государственный электротехнический университет ЛЭТИ}

\ESKDtitle{Пояснительная записка к курсовой работе}


\ESKDsignature{Вариант 3}
\ESKDauthor{Домнин А.В.}
\ESKDchecker{Прокшин А.Н.}
\ESKDdocName{по дисциплине "Информатика"}
\begin{document}
\maketitle
% 1 лист работы
Содержание\\

1 Цель и тема курсовой работы\\

2 Исследование функции\\

3 Исследование кубического сплайна\\

4 Задача оптимального распределения неоднородных ресурсов\\



2 Исследование функции\\

\begin{tikzpicture}[yscale=1,xscale=1]

%Построение осей
\draw[thin,->](-8,0)--(8,0) node [right] {$X$};
\draw[thin,->](0,-2)--(0,5) node [left] {$Y$};
%Построение графика
\draw [domain=-8:8, help lines,thick,smooth, red] plot ({\x},{sqrt(3))*sin(\x r)+cos(\x r)-cos((2*\x r)+pi/3))+1});% node [under,black]{$sqrt3sin(x)+cos(x)-cos(2x+pi/3)+1$};
\node [below] {$0$};\\
\end{tikzpicture}
Слогласно задания функция функция должна быть определена на участке от 0 до 5п/6, на рисунке 1 изображен график функции\\

\begin{tikzpicture}[yscale=2.5,xscale=2.5]

%Построение осей
\draw[thin,->](0,0)--(3,0) node [right] {$X$};
\draw[thin,->](0,-2)--(0,5) node [left] {$Y$};
%Построение графика на участке от 0 до 5п/6
\draw [domain=0:2.616, help lines,thick,smooth, red] plot ({\x},{sqrt(3))*sin(\x r)+cos(\x r)-cos((2*\x r)+pi/3))+1});% node [under,black]{$sqrt3sin(x)+cos(x)-cos(2x+pi/3)+1$};
\node [below] {$0$};
%Построение перпиндикуляра
\draw[thin,black,dashed] (2.616,0.5)--(2.616,0) node [below] {$2.616$};
\end{tikzpicture}\\

Корни уравнения вида sqrt3sin(x)+cos(x)-cos(2x+pi/3)+1 представлены ниже\\
\\
\left\{x=
\begin{array}{cc}
    &0.5236 \\
    &1.1519 \\
    &1.7902 \\
    &2.4086 \\
\end{array}
\right.

\end{document}