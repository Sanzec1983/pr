\documentclass{article}
\usepackage[T2A]{fontenc}
\usepackage[utf8x]{inputenc}
\usepackage{xcolor}
\usepackage[english, russian]{babel}
\usepackage{tikz}
\usepackage[european,cuteinductors, smartlabels]{circuitikz}
\title{Практическая работа №5}
\author{студент гр8871 Домнин А.В } 
% Конец преамбулы
\begin{document}
\maketitle
\newcommand{\D}{10}
\newcommand{\I}{0.6}
\newcommand{\B}{0.2}
Дана функция (ДНФ)
$$
F=B\overline{C}D+A\overline{D}+BC
$$
Необходимо сделать:
\begin{itemize}
    \item таблицу истинности
    \item СДНФ:Для написания формулы по таблице истинности необходимо выписать конъюкции аргументов тех наборов, на которых функция равна 1, причем аргумент равный 0, входит в конъюкцию с отрицанием, а аргумент, равный 1-- без отрицания. Затем следует соединить все образованные конъюкции знаком дизъюнкции.
    \item СКНФ:При составлении формулы {\it по} записываем дизъюнкции аргументов тех наборов, где $F=0$. Аргумент в дизъюнкции входит с отрицанием, если в наборе он равен 1. Все составленные дизъюнкции обьединяем опреацией конъюнкциии
    \item Карты Карно: Прямоугольник делится на равные части столько раз,сколько переменных . Деление осуществляется вертикальным или горизонтальными линиями. Одна половина функции лежит в области, где аргумент равен 0, другая -- где аргумент равен 1. Над областью (или слева от области) где аргумент равен 1, проводится черта и подписывается имя аргумента. Каждый квадрат карты соотвествует набору таблицы.
\end{itemize}
%Составляется таблица истинности



\begin{table}[ht]
\centering
\begin{tabular}{|c|c|c|c|c|} \hline
A & B & C & D & F \\
\hline
0 & 0 & 0 & 0 & 0 \\%1
\hline
1 & 0 & 0 & 0 & 1 \\%2
\hline
1 & 1 & 0 & 0 & 1 \\%3
\hline
1 & 1 & 1 & 0 & 1 \\%4
\hline
1 & 1 & 1 & 1 & 1 \\%5
\hline
0 & 0 & 0 & 1 & 0 \\%6
\hline
0 & 0 & 1 & 1 & 0 \\%7
\hline
0 & 1 & 1 & 1 & 1 \\%8
\hline
1 & 0 & 0 & 1 & 0 \\%9
\hline
0 & 1 & 1 & 0 & 1 \\%10
\hline
1 & 1 & 0 & 1 & 1 \\%11
\hline
1 & 0 & 1 & 1 & 0 \\%12
\hline
1 & 0 & 1 & 0 & 1 \\%13
\hline
0 & 1 & 0 & 1 & 1 \\%14
\hline
0 & 1 & 0 & 0 & 0 \\%15
\hline
0 & 0 & 0 & 1 & 0 \\%16
\hline
\end{tabular}
\label{table}
\caption{Таблица истинности}
\end{table}

Совершенная формула ДНФ будет выглядеть так

\centering
$$
F=A\overline{BCD}\vee AB\overline{CD}\vee ABC\overline{D}\vee ABCD \vee A\overline{BCD} \vee BC\overline{AD} \vee ABD\overline{C} \vee AC\overline{BD}
\vee BD\overline{AC}
$$
Совершенная формула КНФ будет выглядеть так
\centering
$$
F=ABCD \vee ABC\overline{D} \vee AB\overline{CD} \vee BC\overline{AD} \vee B\overline{ACD} \vee ACD\overline{B} \vee ABC\overline{D}
$$
%строится таблица Карно

\begin{table}[ht]
\centering
\begin{tikz}
\draw[thin] (0,0)--({\D*1},{\D*0})
({\D*1},{\D*0})--({\D*1},{\D*(-1)})
({\D*1},{\D*(-1)})--({\D*0},{\D*(-1)})
({\D*0},{\D*(-1)})--({\D\D*0},{\D*0})%Общая таблица


(0,{\D/(-2)})--({\D},{\D/(-2)})
({\D/2},{\D*0})--({\D/2},{\D/(-1)})%Разбиение центральной части

(0,{\D/(-4)})--({\D},{\D/(-4)})
(0,{\D*(-0.75)})--({\D},{\D*(-0.75)})% Горизонтальные
({\D/(4)},0)--({\D/(4)},{\D*(-1)})
({\D*(0.75)},0)--({\D*(0.75)},{\D*(-1})% Вертикальные
%Буквы
({\D/4},0.5) node [below] {${A}$}%A
({\D-\D/8},0.5) node [below] {$\overline{A}$}%A

(-0.5,{\D/(-4)}) node [below] {${C}$}%C
(-0.5,{\D*(-1)+\D/(4)}) node [below] {$\overline{C}$}%C

({\D/8},{\D/(-1)-0.5}) node [below] {${B}$}%B
({\D-\D/8},{\D/(-1)-0.5}) node [below] {$\overline{B}$}%B
({\D/2+\D/8},{\D/(-1)-0.5}) node [below] {${B}$}%B
({\D/4+\D/8},{\D/(-1)-0.5}) node [below] {$\overline{B}$}%B

({\D+0.5},{\D/(-8)}) node [below] {${D}$}%D
({\D+0.5},{\D/(-2)+\D/8}) node [below] {$\overline{D}$}%D
({\D+0.5},{\D/(-2)-\D/8}) node [below] {${D}$}%D
({\D+0.5},{\D/(-1)+\D/(8)}) node [below] {$\overline{D}$}%D
%Цифарки

({\D/8},{\D/(-8)}) node {${15-ABCD}$}%15
({\D/8+\D/(4)},{\D/(-8)}) node {$11-ACD\overline{B}$}%11
({\D/2+\D/(8)},{\D/(-8)}) node {$7-\overline{A}BCD$}%7
({\D-\D/(8))},{\D/(-8)}) node {$3-\overline{AB}CD$}%3

({\D/8},{\D/(-8)+\D/(-4)}) node {$14-ABC\overline{D}$}%14
({\D/8+\D/(4)},{\D/(-8)+\D/(-4)}) node {$10-AC\overline{BD}$}%10
({\D/2+\D/(8)},{\D/(-8)+\D/(-4)}) node {$6-\overline{AD}BC$}%6
({\D-\D/(8))},{\D/(-8)+\D/(-4)}) node {$2-\overline{ABD}C$}%2

({\D/8},{\D*(-0.5)+\D/(-8)}) node {$13-ABD\overline{C}$}%13
({\D/8+\D/(4)},{\D*(-0.5)+\D/(-8)}) node {$9-AD\overline{BC}$}%9
({\D/2+\D/(8)},{\D*(-0.5)+\D/(-8)}) node {$5-\overline{AC}BD$}%5
({\D-\D/(8))},{\D*(-0.5)+\D/(-8)}) node {$1-\overline{ABD}C$}%1

({\D/8},{-\D+\D/8}) node {$12-AB\overline{CD}$}%12
({\D/8+\D/(4)},{-\D+\D/8}) node {$8-A\overline{BCD}$}%8
({\D/2+\D/(8)},{-\D+\D/8}) node {$4-\overline{ACD}B$}%4
({\D-\D/(8))},{-\D+\D/8}) node {$0-\overline{ABCD}$};%0


\end{tikz}
\caption{Карта Карно}
\end{table}
\end{document}