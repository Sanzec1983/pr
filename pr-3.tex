\documentclass{article}
\usepackage[T2A]{fontenc}
\usepackage[utf8x]{inputenc}
\usepackage[english, russian]{babel}
\usepackage{tikz}
\usepackage[european,cuteinductors, smartlabels]{circuitikz}
\title{Практическая работа №3}
\author{студент гр8871 Домнин А.В } 
% Конец преамбулы
\begin{document}

\maketitle

\begin{tikzpicture}[scale=1.4] 

\newcommand{\xa}{2}
\newcommand{\xb}{-2}
\newcommand{\xc}{-3}
%Построение осей
\draw[thin,->](-4,0)--(3,0) node [right] {$Y$};
\draw[thin,->](0,-5)--(0,3) node [left] {$X$};

%Построение графика
\draw [domain=-3.2:2.1, help lines, smooth, red] plot ({\x},{(\x-\xa)*(\x-\xb)*(\x-\xc)}) node [below,black] {$f(x)=x^3+3x^2-4x-12$};

%Построение касательной 1
\draw [domain=-3.3:-2.9, help lines, smooth, yellow] plot ({\x},{(7.52*\x)+22.816}) node [left,black] {$7.52x+22.816$};

%Построение касательной 2
\draw [domain=-2.5:-1, help lines, smooth, purple] plot ({\x},{(-5.08*\x)-10.056}) node [right,black] {$-5.08x-10.056$};

%Построение касательной 3
\draw [domain=1.8:2.2, help lines, smooth, blue] plot ({\x},{(21.83*\x)-43.752}) node [left,black] {$21.83*x-43.752$};

%Построение перпиндикуляров
\draw[thin,blue,dashed] (-3.2,0)--(-3.2,-1.5) node [below] {$X_01=-3.2$};
\draw[thin,blue,dashed] (-1.8,0)--(-1.8,-1) node [right] {$X_02=-1.8$};
\draw[thin,blue,dashed] (2.1,0)--(2.1,1.5) node [right] {$X_03=2.1$};
\end{tikzpicture}

Функция f(x)=x^3+3x^2-4x-12

Проивзодная функции f(x)^/=3x^2+6x-4

Точка Х_01=-3.2

Уравнение касательной

f(x)=7.52x+22.816

Точка Х_02=-1.8

Уравнение касательной

f(x)=-5.08x-10.056

Точка Х_03=2.1

Уравнение касательной

f(x)=21.83*x-43.752









\end{document}
