\documentclass{article}
\usepackage[T2A]{fontenc}
\usepackage[utf8x]{inputenc}
\usepackage[english, russian]{babel}
\usepackage{tikz}
\usepackage[european,cuteinductors, smartlabels]{circuitikz}
\title{Практическая работа №3}
\author{студент гр8871 Домнин А.В } 
% Конец преамбулы
\begin{document}
	
	\maketitle
	
	\begin{tikzpicture}[scale=4,xscale=7]
	
	\newcommand{\xa}{2}
	\newcommand{\xb}{-2}
	\newcommand{\xc}{-3}
	
	%Построение осей
	\draw[thin,->](-3.2,0)--(-2.7,0) node [right] {$X$};
	\draw[thin,->](0,-2.7)--(0,3) node [left] {$Y$};
	
	%Построение графика
	\draw [domain=-3.2:-2.8, help lines,thick,smooth, yellow] plot ({\x},{(\x-\xa)*(\x-\xb)*(\x-\xc)}) node [left,below,black] {$f(x)=x^3+3x^2-4x-12$};
	
	%Построение касательной 1 для 1 корня
	\draw [domain=-3.2:-2.9, help lines, smooth, black] plot ({\x},{(7.52*\x)+22.816}) node [left,black] {$7.52x+22.816$};
	%Построение касательной 1 для 2 корня
	\draw [domain=-3.2:-2.8, help lines, smooth, red] plot ({\x},{(6.23*\x)+18.752}) node [left,red] {$6.23x+18.752$};
	%Построение касательной 1 для 3 корня
	\draw [domain=-3.2:-2.69, help lines, smooth, orange] plot ({\x},{(5.12*\x)+15.361}) node [left,orange] {$5.12x+15.361$};
	%%%%%%%%%%%%%%%
	
	%Построение касательной 2
	%\draw [domain=-2.5:-1, help lines, smooth, purple] plot ({\x},{(-5.08*\x)-10.056}) node [right,black] {$-5.08x-10.056$};
	
	%Построение касательной 3
	%\draw [domain=1.8:2.2, help lines, smooth, blue] plot ({\x},{(21.83*\x)-43.752}) node [left,black] {$21.83*x-43.752$};
	
	%Построение перпиндикуляров
	%Построение перпиндикуляров для 1 корня
	\draw[thin,blue,dashed] (-3.2,0)--(-3.2,-1.25) node [below] {$X_{11}=-3.2$};
	\draw[thin,blue,dashed] (-3.1,0)--(-3.1,-0.58) node [below] {$X_{12}=-3.1$};
	\draw[thin,blue,dashed] (-3.01,0)--(-3.01,-0.055) node [below] {$X_{13}=-3.01$};
	%Построение перпиндикуляров для 2 корня
	%\draw[thin,blue,dashed] (-1.8,0)--(-1.8,-1) node [right] {$X_{21}=-1.8$};
	%Построение перпиндикуляров для 3 корня
	%\draw[thin,blue,dashed] (2.1,0)--(2.1,1.5) node [left] {$X_{31}=2.1$};
	\end{tikzpicture}
	
	Функция f(x)=x^3+3x^2-4x-12
	
	Проивзодная функции f(x)^/=3x^2+6x-4
	
	Точка X_{11}=-3.2
	Точка X_{12}=-3.1
	Точка X_{13}=-3.05
	
	Уравнение касательной
	
	f(x)=7.52x+22.816
	
	Точка X_{21}=-1.8
	
	Уравнение касательной
	
	f(x)=-5.08x-10.056
	
	Точка X_{31}=2.1
	
	Уравнение касательной
	
	f(x)=21.83*x-43.752
	
	
	
	
	
	
	
	
	
\end{document}
