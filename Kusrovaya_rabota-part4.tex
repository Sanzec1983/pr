\documentclass[russian,utf8,nocolumnxxxi,nocolumnxxxii]{eskdtext}
\usepackage[T1,T2A]{fontenc}
\usepackage[utf8]{inputenc}
\usepackage{amssymb,amsmath}
\usepackage{tikz}
\usepackage{siunitx}
\usepackage[american,cuteinductors,smartlabels]{circuitikz}
\usepackage[backend=biber]{biblatex}
\addbibresource{error_estimation_otchet.bib}
\usepackage[]{hyperref}
\hypersetup{
colorlinks=true,
}

\usepackage{textcomp}
\newcommand{\No}{\textnumero}
%Титульный лист
\ESKDdepartment{Федеральное агенство по образованию}

\ESKDcompany{Санкт-Петербургский государственный электротехнический университет ЛЭТИ}

\ESKDtitle{Пояснительная записка к курсовой работе}
\ESKDsignature{Вариант 3}
\ESKDauthor{Домнин А.В.}
\ESKDchecker{Прокшин А.Н.}
\ESKDdocName{по дисциплине "Информатика"}
\begin{document}
\maketitle
% 1 лист работы
Содержание\\

1 Цель и тема курсовой работы\\

2 Исследование функции\\

3 Исследование кубического сплайна\\

4 Задача оптимального распределения неоднородных ресурсов\\
\\
\\
\\
\\
\\
\\
\\
\\
\\
\\
\\
\\
\\
\\
\\
\\
\\
\\
\\
\\
\\
1 Цель курсовой работы: уметь применять персональный компьютер и
математические пакеты прикладных программ в инженерной деятельности.\\
2 Исследование функции\\
a)Решение уравнения вида f(x) = g(x)\\
\begin{equation}
f(x)= \sqrt{3\mathstrut}sin(x) + cos(x)
\end{equation}
\begin{equation}
g(x)=cos\left(2x+\frac n{3} \right)+1
\end{equation}
Пользуясь математическим пакетом Scilab были получены следующие корни уравнения на интервале от -10 до 10\\
\begin{equation}
x=\left\{
\begin{array}{c}

    &-6.807
    &-3.665
    &-0.524
    &2.618
    &5.76

    \end{array}
\right.
\end{equation}
б)Исследование функции на промежутке от 0 до 5п/6\\
На рисунке 1 изображена функция на интервале от -7 до 7\\ 
\begin{figure}[!ht]
    \centering
\begin{tikzpicture}[yscale=1,xscale=1]
%Построение осей
\draw[thin,->](-7,0)--(7,0) node [right] {$X$};
\draw[thin,->](0,-2)--(0,5) node [left] {$Y$};
%Построение графика
\draw [domain=-7:7, help lines,thick,smooth, red] plot ({\x},{sqrt(3)*sin(\x r)+cos(\x r)-cos((2*\x r)+(pi/3 r))+1});
%подпись 0
\node [below] {$0$};
\draw (5,6)node [black]{$\sqrt{3\mathstrut}\cdot sin(x)+cos(x)-cos\left(2x+\frac n{3}\right)+1$};
\end{tikzpicture} 
    \caption{Построение графика функции}
    \label{fig:my_label}
\end{figure}

Слогласно задания функция функция должна быть определена на участке от 0 до 5п/6, на рисунке 2 изображен график функции
\begin{figure}[!ht]
    \centering
\begin{tikzpicture}[yscale=2.5,xscale=2.5]
%Построение осей
\draw[thin,->](0,0)--(3,0) node [right] {$X$};
\draw[thin,->](0,-0.2)--(0,5) node [left] {$Y$};
%Построение графика на участке от 0 до 5п/6
\draw [domain=0:2.616, help lines,thick,smooth, red] plot ({\x},{sqrt(3)*sin(\x r)+cos(\x r)-cos((2*\x r)+(pi/3 r))+1});
\node [below] {$0$};
%Построение перпиндикуляра,точки максимума
\draw[thin,black,dashed] (2.618,0.5)--(2.618,0) node [below] {$2.618$};
\draw (1.05,0) node [below] {$1.05$};%точка максимума по Х
\draw (0,4) node [left] {$4$};%точка максимума по У
\draw[thin,black,dashed] (1.05,0)--(1.05,4);%построние вертикали
\draw[thin,black,dashed] (0,4)--(1.05,4);%построние горизонтали
\end{tikzpicture}
\caption{Построение графика функции на ограниченном участке}
    \label{fig:my_label}
\end{figure}
Корни уравнения вида \\
$\sqrt{3\mathstrut}\cdot sin(x)+cos(x)-cos\left(2x+\frac n{3}\right)+1$ представлены ниже
\begin{equation}
x=2.618
\end{equation}
На участке от 0 до 5п/6 функция имеет один "0" и он находится в точке 2,618 на иллюстрирует график. Максимум находится в точке x=1.05, y=4.
1-я производная функции равна
\begin{equation}
\sqrt{3\mathstrut}\cdotcos(x)+2sin\left(\frac{n+6x}{3}\right)-sin(x)
\end{equation}
График производной приведен на рисунке 3 % 1 производная
\begin{figure}[!ht]
    \centering
\begin{tikzpicture}[yscale=1,xscale=1]
%Построение осей
\draw[thin,->](-5,0)--(5,0) node [right] {$X$};
\draw[thin,->](0,-5)--(0,5) node [left] {$Y$};
%Построение графика производной на всем участке 
\draw [domain=-5:5, help lines,thick,smooth, red] plot ({\x},{sqrt(3)*cos(\x r)+2*sin((pi r+6*\x r)/3)-sin(\x r)});
\node [below] {$0$};
\end{tikzpicture}
\caption{Построение графика производной}
    \label{fig:my_label}
\end{figure}
%Построение графика производной на участке от 0 до 5п/6
\begin{figure}[!ht]
    \centering
\begin{tikzpicture}[yscale=1,xscale=1]
%Построение осей
\draw[thin,->](0,0)--(2.9,0) node [right] {$X$};
\draw[thin,->](0,0)--(0,5) node [left] {$Y$};
%производная на участке
\draw [domain=0:2.618, help lines,thick,smooth, red] plot (({\x},{sqrt(3)*cos(\x r)+2*sin((pi r+6*\x r)/3)-sin(\x r)});
\node [left] {$0$};
%Построение перпиндикуляра,точек максимума и минимума
\draw[thin,black,dashed] (0,3.52)--(0.11,3.52) node [left] {$3.52$};%точка максимума по У
\draw[thin,black,dashed] (0.11,3.52)--(0.11,0) node [right,below] {$0.11$};%точка максимума по Х
\draw (1.05,0) node [right,above] {$1.05$};%ноль
\draw[thin,black,dashed] (1.98,0)--(1.98,-3.52);%минимум по У
\draw (1.98,0) node [right,below] {$1.98$};
\draw (1.98,-3.52) node [right,below] {$-3.52$};
\end{tikzpicture}
\caption{Построение графика функции на ограниченном участке}
    \label{fig:my_label}
\end{figure}

%%%%%%% 2 часть работы

\begin{figure}[!ht]
    \centering
\begin{tikzpicture}[yscale=1,xscale=3]

%Построение осей
\draw[thin,->](0,0)--(3.3,0) node [right] {$X$};
\draw[thin,->](0,0)--(0,4.6) node [left] {$Y$};
\node [below] {$0$};
%построение точек
\draw [ball color=black]
(0,4) circle (0.02) node [left] {$1$};
\draw [ball color=black]
(0.25,3.6) circle (0.02) node [above] {$2$};
\draw [ball color=black]
(1.25,4.575) circle (0.02) node [left] {$3$};
\draw [ball color=black]
(2.125,4.017) circle (0.02) node [left] {$4$};
\draw [ball color=black]
(3.25,3.83) circle (0.02) node [left] {$5$};
\end{tikzpicture}
\caption{Расположение точек на плоскости}
    \label{fig:my_label}
\end{figure}
3 Исследование кубического сплайна
Для того чтобы потенциальная энергия изогнутой металлической линейки(сплайна) принимала минимальное значение,производная четвертого порядка должна быть равна нулю, следовательно можно представить сплайн полиномом третьей степени на каждом отрезке [$xi, x_{i+1}$]
Уравнение сплайна на 1 участке
\begin{equation}
P_1(x) =A_{10}+A_{11}x_1+A_{12}x^2_1+A_{13}x^3_1
\end{equation}
\begin{equation}
P_2(x) =A_{10}+A_{11}x_2+A_{12}x^2_2+A_{13}x^3_2
\end{equation}
\begin{equation}
P_2(x) =A_{20}+A_{21}x_2+A_{22}x^2_2+A_{23}x^3_2
\end{equation}
\begin{equation}
P_3(x) =A_{20}+A_{21}x_3+A_{22}x^2_3+A_{23}x^3_3
\end{equation}
\begin{equation}
P_3(x) =A_{30}+A_{31}x_3+A_{32}x^2_3+A_{33}x^3_3
\end{equation}
\begin{equation}
P_4(x) =A_{30}+A_{31}x_4+A_{62}x^2_4+A_{63}x^3_4
\end{equation}

Производные во внутренних точках

\begin{equation}
A_{11} + 2A_{12}x_2 + 3A_{43}x^2_2=A_{21} + 2A_{22}x_2 + 3A_{23}x^2_2
\end{equation}
\begin{equation}
A_{21} + 2A_{22}x_3 + 3A_{23}x^2_3= A_{31} + 2A_{32}x_2 + 3A_{33}x^2_3
\end{equation}
Производные второго порядка в точках склейки
\begin{equation}
2A_{12} + 6A_{13}x_2=2A_{22} + 6A_{23}x_2
\end{equation}
\begin{equation}
2A_{22} + 6A_{23}x_3= 2A_{32} + 6A_{33}x_3
\end{equation}
\end{document}