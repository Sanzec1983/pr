\documentclass{beamer}
\usepackage[T1,T2A]{fontenc}
\usepackage[utf8]{inputenc} 
\usepackage[english,russian]{babel}
\usepackage{tikz}
\usetheme{Madrid}
\usepackage{csquotes}
\newcommand{\quotes}[1]{``#1''}
%https://www.overleaf.com/help/107-how-to-create-a-basic-slideshow-presentation-in-latex-with-beamer#.V8mZiNFEFqM


\title{Поворот тела в пространстве}
\subtitle{Поворот на углы по осям}
\author{Домнин Александр Вячеславович} 


\begin{document}
\begin{frame}
\titlepage
\end{frame}

\begin{frame}
  \tableofcontents
\end{frame}



%%1
\section{Положение в нулевом положении}
\begin{frame}
\frametitle{\small Положение в нулевом положении}
\end{frame}
\begin{tikzpicture}
\newcommand{\rr}{0.2}
\newcommand{\xa}{0}%1
\newcommand{\ya}{0}%1
\newcommand{\xb}{0}%2
\newcommand{\yb}{0}%2
\newcommand{\xc}{6}%3
\newcommand{\yc}{0}%3
\newcommand{\xd}{6}%4
\newcommand{\yd}{0}%4
\newcommand{\xe}{6}%5
\newcommand{\ye}{6}%5
\newcommand{\xf}{6}%6
\newcommand{\yf}{6}%6

%построение 1 шарика

\draw [rounded corners=14pt,color=white, ball color=black,smooth] ({\xa},{\ya}) circle ({\rr}) node {1};
%построение 2 шарика
\draw [rounded corners=14pt,color=white, ball color=black,smooth] ({\xb},{\yb}) circle ({\rr}) node {2};
%построение 3 шарика
\draw [rounded corners=14pt,color=white, ball color=black,smooth] ({\xc},{\yc}) circle ({\rr}) node {3};
%построение 4 шарика 
\draw [rounded corners=14pt,color=white, ball color=black,smooth] ({\xd},{\yd}) circle ({\rr}) node {4};
%построение 5 шарика 
\draw [rounded corners=14pt,color=white, ball color=black,smooth] ({\xe},{\ye}) circle ({\rr}) node {5};
%построение 6 шарика 
\draw [rounded corners=14pt,color=white, ball color=black,smooth] ({\xf},{\yf}) circle ({\rr}) node {6};

%построение граней

%построение грани 1
\draw[line width=6,black]  ({\xa},{\ya})--({\xb},{\yb});
%построение грани 2
\draw[line width=6,black]  ({\xb},{\yb})--({\xc},{\yc});
%построение грани 3
\draw[line width=6,black]  ({\xa},{\ya})--({\xd},{\yd});
%построение грани 4
\draw[line width=6,black]  ({\xc},{\yc})--({\xd},{\yd});
%построение грани 7
\draw[line width=6,black]  ({\xa},{\ya})--({\xd},{\yd});
%построение грани 8
\draw[line width=6,black]  ({\xc},{\yc})--({\xf},{\yf});
%построение грани 9
\draw[line width=6,black]  ({\xd},{\yd})--({\xe},{\ye});
%построение грани 10
\draw[line width=6,black]  ({\xe},{\ye})--({\xf},{\yf});

\end{tikzpicture}


%%% 2

\section{Положение обьекта  при повороте на 15 градусов по каждой оси}
\begin{frame}
\frametitle{\small Положение обьекта  при повороте на15 градусов по каждой оси}
\end{frame}
%\tableofcontents[Положение обьекта  при повороте на 15  градусов по каждой оси]

\begin{tikzpicture}

\newcommand{\rr}{0.2}
\newcommand{\xa}{0}%1
\newcommand{\ya}{0}%1
\newcommand{\xb}{- 1.8508125}%2
\newcommand{\yb}{1.1117714}%2
\newcommand{\xc}{3.7472637}%3
\newcommand{\yc}{2.6117714}%3
\newcommand{\xd}{5.5980762}%4
\newcommand{\yd}{1.5}%4
\newcommand{\xe}{4.4863048}%5
\newcommand{\ye}{7.2021017}%5
\newcommand{\xf}{2.6354923}%6
\newcommand{\yf}{8.3138732}%6

%построение 1 шарика

\draw [rounded corners=14pt,color=white, ball color=black,smooth] ({\xa},{\ya}) circle ({\rr}) node {1};
%построение 2 шарика
\draw [rounded corners=14pt,color=white, ball color=black,smooth] ({\xb},{\yb}) circle ({\rr}) node {2};
%построение 3 шарика
\draw [rounded corners=14pt,color=white, ball color=black,smooth] ({\xc},{\yc}) circle ({\rr}) node {3};
%построение 4 шарика 
\draw [rounded corners=14pt,color=white, ball color=black,smooth] ({\xd},{\yd}) circle ({\rr}) node {4};
%построение 5 шарика 
\draw [rounded corners=14pt,color=white, ball color=black,smooth] ({\xe},{\ye}) circle ({\rr}) node {5};
%построение 6 шарика 
\draw [rounded corners=14pt,color=white, ball color=black,smooth] ({\xf},{\yf}) circle ({\rr}) node {6};

%построение граней

%построение грани 1
\draw[line width=6,black]  ({\xa},{\ya})--({\xb},{\yb});
%построение грани 2
\draw[line width=6,black]  ({\xb},{\yb})--({\xc},{\yc});
%построение грани 3
\draw[line width=6,black]  ({\xa},{\ya})--({\xd},{\yd});
%построение грани 4
\draw[line width=6,black]  ({\xc},{\yc})--({\xd},{\yd});
%построение грани 7
\draw[line width=6,black]  ({\xa},{\ya})--({\xd},{\yd});
%построение грани 8
\draw[line width=6,black]  ({\xc},{\yc})--({\xf},{\yf});
%построение грани 9
\draw[line width=6,black]  ({\xd},{\yd})--({\xe},{\ye});
%построение грани 10
\draw[line width=6,black]  ({\xe},{\ye})--({\xf},{\yf});

\end{tikzpicture}

%%3

\section{Положение обьекта  при повороте на 30 градусов по каждой оси}
\begin{frame}
\frametitle{\small Положение обьекта  при повороте на 30 градусов по каждой оси}
\end{frame}
%\tableofcontents[Положение обьекта  при повороте на 30 градусов по каждой оси]

\begin{tikzpicture}

\newcommand{\rr}{0.2}
\newcommand{\xa}{0}%1
\newcommand{\ya}{0}%1
\newcommand{\xb}{- 3.75}%2
\newcommand{\yb}{1.2990381}%2
\newcommand{\xc}{0.75}%3
\newcommand{\yc}{3.89711434}%3
\newcommand{\xd}{4.5}%4
\newcommand{\yd}{2.5980762}%4
\newcommand{\xe}{3.2009619}%5
\newcommand{\ye}{7.8480762}%5
\newcommand{\xf}{- 0.5490381}%6
\newcommand{\yf}{9.1471143}%6

%построение 1 шарика

\draw [rounded corners=14pt,color=white, ball color=black,smooth] ({\xa},{\ya}) circle ({\rr}) node {1};
%построение 2 шарика
\draw [rounded corners=14pt,color=white, ball color=black,smooth] ({\xb},{\yb}) circle ({\rr}) node {2};
%построение 3 шарика
\draw [rounded corners=14pt,color=white, ball color=black,smooth] ({\xc},{\yc}) circle ({\rr}) node {3};
%построение 4 шарика 
\draw [rounded corners=14pt,color=white, ball color=black,smooth] ({\xd},{\yd}) circle ({\rr}) node {4};
%построение 5 шарика 
\draw [rounded corners=14pt,color=white, ball color=black,smooth] ({\xe},{\ye}) circle ({\rr}) node {5};
%построение 6 шарика 
\draw [rounded corners=14pt,color=white, ball color=black,smooth] ({\xf},{\yf}) circle ({\rr}) node {6};

%построение граней

%построение грани 1
\draw[line width=6,black]  ({\xa},{\ya})--({\xb},{\yb});
%построение грани 2
\draw[line width=6,black]  ({\xb},{\yb})--({\xc},{\yc});
%построение грани 3
\draw[line width=6,black]  ({\xa},{\ya})--({\xd},{\yd});
%построение грани 4
\draw[line width=6,black]  ({\xc},{\yc})--({\xd},{\yd});
%построение грани 7
\draw[line width=6,black]  ({\xa},{\ya})--({\xd},{\yd});
%построение грани 8
\draw[line width=6,black]  ({\xc},{\yc})--({\xf},{\yf});
%построение грани 9
\draw[line width=6,black]  ({\xd},{\yd})--({\xe},{\ye});
%построение грани 10
\draw[line width=6,black]  ({\xe},{\ye})--({\xf},{\yf});

\end{tikzpicture}

%%4

\section{Положение обьекта  при повороте на 45 градусов по каждой оси}
\begin{frame}
\frametitle{\small Положение обьекта  при повороте на 45 градусов по каждой оси}
\end{frame}
%\tableofcontents[Положение обьекта  при повороте на 45 градусов по каждой оси]

\begin{tikzpicture}

\newcommand{\rr}{0.2}
\newcommand{\xa}{0}%1
\newcommand{\ya}{0}%1
\newcommand{\xb}{- 5.1213203}%2
\newcommand{\yb}{0.8786797}%2
\newcommand{\xc}{- 2.1213203}%3
\newcommand{\yc}{3.8786797}%3
\newcommand{\xd}{3}%4
\newcommand{\yd}{3}%4
\newcommand{\xe}{2.1213203}%5
\newcommand{\ye}{8.1213203}%5
\newcommand{\xf}{-3}%6
\newcommand{\yf}{9}%6

%построение 1 шарика

\draw [rounded corners=14pt,color=white, ball color=black,smooth] ({\xa},{\ya}) circle ({\rr}) node {1};
%построение 2 шарика
\draw [rounded corners=14pt,color=white, ball color=black,smooth] ({\xb},{\yb}) circle ({\rr}) node {2};
%построение 3 шарика
\draw [rounded corners=14pt,color=white, ball color=black,smooth] ({\xc},{\yc}) circle ({\rr}) node {3};
%построение 4 шарика 
\draw [rounded corners=14pt,color=white, ball color=black,smooth] ({\xd},{\yd}) circle ({\rr}) node {4};
%построение 5 шарика 
\draw [rounded corners=14pt,color=white, ball color=black,smooth] ({\xe},{\ye}) circle ({\rr}) node {5};
%построение 6 шарика 
\draw [rounded corners=14pt,color=white, ball color=black,smooth] ({\xf},{\yf}) circle ({\rr}) node {6};

%построение граней

%построение грани 1
\draw[line width=6,black]  ({\xa},{\ya})--({\xb},{\yb});
%построение грани 2
\draw[line width=6,black]  ({\xb},{\yb})--({\xc},{\yc});
%построение грани 3
\draw[line width=6,black]  ({\xa},{\ya})--({\xd},{\yd});
%построение грани 4
\draw[line width=6,black]  ({\xc},{\yc})--({\xd},{\yd});
%построение грани 7
\draw[line width=6,black]  ({\xa},{\ya})--({\xd},{\yd});
%построение грани 8
\draw[line width=6,black]  ({\xc},{\yc})--({\xf},{\yf});
%построение грани 9
\draw[line width=6,black]  ({\xd},{\yd})--({\xe},{\ye});
%построение грани 10
\draw[line width=6,black]  ({\xe},{\ye})--({\xf},{\yf});

\end{tikzpicture}

%%5

\section{Положение обьекта  при повороте на 60 градусов по каждой оси}
\begin{frame}
\frametitle{\small Положение обьекта  при повороте на 60 градусов по каждой оси}
\end{frame}
%\tableofcontents[Положение обьекта  при повороте на 60 градусов по каждой оси]

\begin{tikzpicture}

\newcommand{\rr}{0.2}
\newcommand{\xa}{0}%1
\newcommand{\ya}{0}%1
\newcommand{\xb}{- 5.7990381}%2
\newcommand{\yb}{0.3480762}%2
\newcommand{\xc}{- 4.2990381}%3
\newcommand{\yc}{2.9461524 }%3
\newcommand{\xd}{1.5}%4
\newcommand{\yd}{2.5980762}%4
\newcommand{\xe}{1.1519238}%5
\newcommand{\ye}{7.9951905}%5
\newcommand{\xf}{- 4.6471143}%6
\newcommand{\yf}{8.3432667}%6

%построение 1 шарика

\draw [rounded corners=14pt,color=white, ball color=black,smooth] ({\xa},{\ya}) circle ({\rr}) node {1};
%построение 2 шарика
\draw [rounded corners=14pt,color=white, ball color=black,smooth] ({\xb},{\yb}) circle ({\rr}) node {2};
%построение 3 шарика
\draw [rounded corners=14pt,color=white, ball color=black,smooth] ({\xc},{\yc}) circle ({\rr}) node {3};
%построение 4 шарика 
\draw [rounded corners=14pt,color=white, ball color=black,smooth] ({\xd},{\yd}) circle ({\rr}) node {4};
%построение 5 шарика 
\draw [rounded corners=14pt,color=white, ball color=black,smooth] ({\xe},{\ye}) circle ({\rr}) node {5};
%построение 6 шарика 
\draw [rounded corners=14pt,color=white, ball color=black,smooth] ({\xf},{\yf}) circle ({\rr}) node {6};

%построение граней

%построение грани 1
\draw[line width=6,black]  ({\xa},{\ya})--({\xb},{\yb});
%построение грани 2
\draw[line width=6,black]  ({\xb},{\yb})--({\xc},{\yc});
%построение грани 3
\draw[line width=6,black]  ({\xa},{\ya})--({\xd},{\yd});
%построение грани 4
\draw[line width=6,black]  ({\xc},{\yc})--({\xd},{\yd});
%построение грани 7
\draw[line width=6,black]  ({\xa},{\ya})--({\xd},{\yd});
%построение грани 8
\draw[line width=6,black]  ({\xc},{\yc})--({\xf},{\yf});
%построение грани 9
\draw[line width=6,black]  ({\xd},{\yd})--({\xe},{\ye});
%построение грани 10
\draw[line width=6,black]  ({\xe},{\ye})--({\xf},{\yf});

\end{tikzpicture}

%%6

\section{Положение обьекта  при повороте на 75 градусов по каждой оси}
\begin{frame}
\frametitle{\small Положение обьекта  при повороте на 75 градусов по каждой оси}
\end{frame}
%\tableofcontents[Положение обьекта  при повороте на 75 градусов по каждой оси]

\begin{tikzpicture}

\newcommand{\rr}{0.2}
\newcommand{\xa}{0}%1
\newcommand{\ya}{0}%1
\newcommand{\xb}{- 5.9863048}%2
\newcommand{\yb}{0.0511113}%2
\newcommand{\xc}{- 5.584381}%3
\newcommand{\yc}{1.5511113}%3
\newcommand{\xd}{0.4019238}%4
\newcommand{\yd}{1.5}%4
\newcommand{\xe}{0.3508125}%5
\newcommand{\ye}{7.3092502}%5
\newcommand{\xf}{- 5.6354923}%6
\newcommand{\yf}{7.3603614}%6

%построение 1 шарика

\draw [rounded corners=14pt,color=white, ball color=black,smooth] ({\xa},{\ya}) circle ({\rr}) node {1};
%построение 2 шарика
\draw [rounded corners=14pt,color=white, ball color=black,smooth] ({\xb},{\yb}) circle ({\rr}) node {2};
%построение 3 шарика
\draw [rounded corners=14pt,color=white, ball color=black,smooth] ({\xc},{\yc}) circle ({\rr}) node {3};
%построение 4 шарика 
\draw [rounded corners=14pt,color=white, ball color=black,smooth] ({\xd},{\yd}) circle ({\rr}) node {4};
%построение 5 шарика 
\draw [rounded corners=14pt,color=white, ball color=black,smooth] ({\xe},{\ye}) circle ({\rr}) node {5};
%построение 6 шарика 
\draw [rounded corners=14pt,color=white, ball color=black,smooth] ({\xf},{\yf}) circle ({\rr}) node {6};

%построение граней

%построение грани 1
\draw[line width=6,black]  ({\xa},{\ya})--({\xb},{\yb});
%построение грани 2
\draw[line width=6,black]  ({\xb},{\yb})--({\xc},{\yc});
%построение грани 3
\draw[line width=6,black]  ({\xa},{\ya})--({\xd},{\yd});
%построение грани 4
\draw[line width=6,black]  ({\xc},{\yc})--({\xd},{\yd});
%построение грани 7
\draw[line width=6,black]  ({\xa},{\ya})--({\xd},{\yd});
%построение грани 8
\draw[line width=6,black]  ({\xc},{\yc})--({\xf},{\yf});
%построение грани 9
\draw[line width=6,black]  ({\xd},{\yd})--({\xe},{\ye});
%построение грани 10
\draw[line width=6,black]  ({\xe},{\ye})--({\xf},{\yf});

\end{tikzpicture}

%%7

\section{Положение обьекта  при повороте на 90 градусов по каждой оси}
\begin{frame}
\frametitle{\small Положение обьекта  при повороте на 90 градусов по каждой оси}
\end{frame}
%\tableofcontents[Положение обьекта  при повороте на 90 градусов по каждой оси]

\begin{tikzpicture}

\newcommand{\rr}{0.2}
\newcommand{\xa}{0}%1
\newcommand{\ya}{0}%1
\newcommand{\xb}{-6}%2
\newcommand{\yb}{0}%2
\newcommand{\xc}{-6}%3
\newcommand{\yc}{0}%3
\newcommand{\xd}{0}%4
\newcommand{\yd}{0}%4
\newcommand{\xe}{0}%5
\newcommand{\ye}{6}%5
\newcommand{\xf}{-6}%6
\newcommand{\yf}{6}%6

%построение 1 шарика

\draw [rounded corners=14pt,color=white, ball color=black,smooth] ({\xa},{\ya}) circle ({\rr}) node {1};
%построение 2 шарика
\draw [rounded corners=14pt,color=white, ball color=black,smooth] ({\xb},{\yb}) circle ({\rr}) node {2};
%построение 3 шарика
\draw [rounded corners=14pt,color=white, ball color=black,smooth] ({\xc},{\yc}) circle ({\rr}) node {3};
%построение 4 шарика 
\draw [rounded corners=14pt,color=white, ball color=black,smooth] ({\xd},{\yd}) circle ({\rr}) node {4};
%построение 5 шарика 
\draw [rounded corners=14pt,color=white, ball color=black,smooth] ({\xe},{\ye}) circle ({\rr}) node {5};
%построение 6 шарика 
\draw [rounded corners=14pt,color=white, ball color=black,smooth] ({\xf},{\yf}) circle ({\rr}) node {6};

%построение граней

%построение грани 1
\draw[line width=6,black]  ({\xa},{\ya})--({\xb},{\yb});
%построение грани 2
\draw[line width=6,black]  ({\xb},{\yb})--({\xc},{\yc});
%построение грани 3
\draw[line width=6,black]  ({\xa},{\ya})--({\xd},{\yd});
%построение грани 4
\draw[line width=6,black]  ({\xc},{\yc})--({\xd},{\yd});
%построение грани 7
\draw[line width=6,black]  ({\xa},{\ya})--({\xd},{\yd});
%построение грани 8
\draw[line width=6,black]  ({\xc},{\yc})--({\xf},{\yf});
%построение грани 9
\draw[line width=6,black]  ({\xd},{\yd})--({\xe},{\ye});
%построение грани 10
\draw[line width=6,black]  ({\xe},{\ye})--({\xf},{\yf});

\end{tikzpicture}

%%8

\section{Положение обьекта  при повороте на 105 градусов по каждой оси}
\begin{frame}
\frametitle{\small Положение обьекта  при повороте на 105 градусов по каждой оси}
\end{frame}
%\tableofcontents[Положение обьекта  при повороте на 105 градусов по каждой оси]

\begin{tikzpicture}

\newcommand{\rr}{0.2}
\newcommand{\xa}{0}%1
\newcommand{\ya}{0}%1
\newcommand{\xb}{- 5.9863048}%2
\newcommand{\yb}{- 0.0511113}%2
\newcommand{\xc}{- 5.584381}%3
\newcommand{\yc}{- 1.5511113}%3
\newcommand{\xd}{0.4019238}%4
\newcommand{\yd}{- 1.5 }%4
\newcommand{\xe}{0.4530350}%5
\newcommand{\ye}{4.3092502}%5
\newcommand{\xf}{- 5.5332697}%6
\newcommand{\yf}{4.2581389}%6

%построение 1 шарика

\draw [rounded corners=14pt,color=white, ball color=black,smooth] ({\xa},{\ya}) circle ({\rr}) node {1};
%построение 2 шарика
\draw [rounded corners=14pt,color=white, ball color=black,smooth] ({\xb},{\yb}) circle ({\rr}) node {2};
%построение 3 шарика
\draw [rounded corners=14pt,color=white, ball color=black,smooth] ({\xc},{\yc}) circle ({\rr}) node {3};
%построение 4 шарика 
\draw [rounded corners=14pt,color=white, ball color=black,smooth] ({\xd},{\yd}) circle ({\rr}) node {4};
%построение 5 шарика 
\draw [rounded corners=14pt,color=white, ball color=black,smooth] ({\xe},{\ye}) circle ({\rr}) node {5};
%построение 6 шарика 
\draw [rounded corners=14pt,color=white, ball color=black,smooth] ({\xf},{\yf}) circle ({\rr}) node {6};

%построение граней

%построение грани 1
\draw[line width=6,black]  ({\xa},{\ya})--({\xb},{\yb});
%построение грани 2
\draw[line width=6,black]  ({\xb},{\yb})--({\xc},{\yc});
%построение грани 3
\draw[line width=6,black]  ({\xa},{\ya})--({\xd},{\yd});
%построение грани 4
\draw[line width=6,black]  ({\xc},{\yc})--({\xd},{\yd});
%построение грани 7
\draw[line width=6,black]  ({\xa},{\ya})--({\xd},{\yd});
%построение грани 8
\draw[line width=6,black]  ({\xc},{\yc})--({\xf},{\yf});
%построение грани 9
\draw[line width=6,black]  ({\xd},{\yd})--({\xe},{\ye});
%построение грани 10
\draw[line width=6,black]  ({\xe},{\ye})--({\xf},{\yf});

\end{tikzpicture}

%%9

\section{Положение обьекта  при повороте на 120 градусов по каждой оси}
\begin{frame}
\frametitle{\small Положение обьекта  при повороте на 120 градусов по каждой оси}
\end{frame}
%\tableofcontents[Положение обьекта  при повороте на 120 градусов по каждой оси]

\begin{tikzpicture}

\newcommand{\rr}{0.2}
\newcommand{\xa}{0}%1
\newcommand{\ya}{0}%1
\newcommand{\xb}{- 5.7990381}%2
\newcommand{\yb}{- 0.3480762}%2
\newcommand{\xc}{- 4.2990381}%3
\newcommand{\yc}{- 2.9461524}%3
\newcommand{\xd}{1.5}%4
\newcommand{\yd}{- 2.5980762}%4
\newcommand{\xe}{1.8480762}%5
\newcommand{\ye}{2.7990381}%5
\newcommand{\xf}{- 3.9509619}%6
\newcommand{\yf}{2.4509619}%6

%построение 1 шарика

\draw [rounded corners=14pt,color=white, ball color=black,smooth] ({\xa},{\ya}) circle ({\rr}) node {1};
%построение 2 шарика
\draw [rounded corners=14pt,color=white, ball color=black,smooth] ({\xb},{\yb}) circle ({\rr}) node {2};
%построение 3 шарика
\draw [rounded corners=14pt,color=white, ball color=black,smooth] ({\xc},{\yc}) circle ({\rr}) node {3};
%построение 4 шарика 
\draw [rounded corners=14pt,color=white, ball color=black,smooth] ({\xd},{\yd}) circle ({\rr}) node {4};
%построение 5 шарика 
\draw [rounded corners=14pt,color=white, ball color=black,smooth] ({\xe},{\ye}) circle ({\rr}) node {5};
%построение 6 шарика 
\draw [rounded corners=14pt,color=white, ball color=black,smooth] ({\xf},{\yf}) circle ({\rr}) node {6};

%построение граней

%построение грани 1
\draw[line width=6,black]  ({\xa},{\ya})--({\xb},{\yb});
%построение грани 2
\draw[line width=6,black]  ({\xb},{\yb})--({\xc},{\yc});
%построение грани 3
\draw[line width=6,black]  ({\xa},{\ya})--({\xd},{\yd});
%построение грани 4
\draw[line width=6,black]  ({\xc},{\yc})--({\xd},{\yd});
%построение грани 7
\draw[line width=6,black]  ({\xa},{\ya})--({\xd},{\yd});
%построение грани 8
\draw[line width=6,black]  ({\xc},{\yc})--({\xf},{\yf});
%построение грани 9
\draw[line width=6,black]  ({\xd},{\yd})--({\xe},{\ye});
%построение грани 10
\draw[line width=6,black]  ({\xe},{\ye})--({\xf},{\yf});

\end{tikzpicture}

%%10

\section{Положение обьекта  при повороте на 135 градусов по каждой оси}
\begin{frame}
\frametitle{\small Положение обьекта  при повороте на 135 градусов по каждой оси}
\end{frame}
%\tableofcontents[Положение обьекта  при повороте на 135 градусов по каждой оси]

\begin{tikzpicture}

\newcommand{\rr}{0.2}
\newcommand{\xa}{0}%1
\newcommand{\ya}{0}%1
\newcommand{\xb}{- 5.1213203}%2
\newcommand{\yb}{- 0.8786797}%2
\newcommand{\xc}{- 2.1213203}%3
\newcommand{\yc}{- 3.8786797}%3
\newcommand{\xd}{3}%4
\newcommand{\yd}{-3}%4
\newcommand{\xe}{3.8786797}%5
\newcommand{\ye}{2.1213203}%5
\newcommand{\xf}{- 1.2426407}%6
\newcommand{\yf}{1.2426407}%6

%построение 1 шарика

\draw [rounded corners=14pt,color=white, ball color=black,smooth] ({\xa},{\ya}) circle ({\rr}) node {1};
%построение 2 шарика
\draw [rounded corners=14pt,color=white, ball color=black,smooth] ({\xb},{\yb}) circle ({\rr}) node {2};
%построение 3 шарика
\draw [rounded corners=14pt,color=white, ball color=black,smooth] ({\xc},{\yc}) circle ({\rr}) node {3};
%построение 4 шарика 
\draw [rounded corners=14pt,color=white, ball color=black,smooth] ({\xd},{\yd}) circle ({\rr}) node {4};
%построение 5 шарика 
\draw [rounded corners=14pt,color=white, ball color=black,smooth] ({\xe},{\ye}) circle ({\rr}) node {5};
%построение 6 шарика 
\draw [rounded corners=14pt,color=white, ball color=black,smooth] ({\xf},{\yf}) circle ({\rr}) node {6};

%построение граней

%построение грани 1
\draw[line width=6,black]  ({\xa},{\ya})--({\xb},{\yb});
%построение грани 2
\draw[line width=6,black]  ({\xb},{\yb})--({\xc},{\yc});
%построение грани 3
\draw[line width=6,black]  ({\xa},{\ya})--({\xd},{\yd});
%построение грани 4
\draw[line width=6,black]  ({\xc},{\yc})--({\xd},{\yd});
%построение грани 7
\draw[line width=6,black]  ({\xa},{\ya})--({\xd},{\yd});
%построение грани 8
\draw[line width=6,black]  ({\xc},{\yc})--({\xf},{\yf});
%построение грани 9
\draw[line width=6,black]  ({\xd},{\yd})--({\xe},{\ye});
%построение грани 10
\draw[line width=6,black]  ({\xe},{\ye})--({\xf},{\yf});

\end{tikzpicture}

%%11

\section{Положение обьекта  при повороте на 150 градусов по каждой оси}
\begin{frame}
\frametitle{\small Положение обьекта  при повороте на 150 градусов по каждой оси}
\end{frame}
%\tableofcontents[Положение обьекта  при повороте на 150 градусов по каждой оси]

\begin{tikzpicture}

\newcommand{\rr}{0.2}
\newcommand{\xa}{0}%1
\newcommand{\ya}{0}%1
\newcommand{\xb}{- 3.75}%2
\newcommand{\yb}{- 1.2990381}%2
\newcommand{\xc}{0.75}%3
\newcommand{\yc}{- 3.8971143}%3
\newcommand{\xd}{4.5}%4
\newcommand{\yd}{- 2.5980762}%4
\newcommand{\xe}{5.7990381}%5
\newcommand{\ye}{2.6519238}%5
\newcommand{\xf}{2.0490381}%6
\newcommand{\yf}{1.3528857 }%6

%построение 1 шарика

\draw [rounded corners=14pt,color=white, ball color=black,smooth] ({\xa},{\ya}) circle ({\rr}) node {1};
%построение 2 шарика
\draw [rounded corners=14pt,color=white, ball color=black,smooth] ({\xb},{\yb}) circle ({\rr}) node {2};
%построение 3 шарика
\draw [rounded corners=14pt,color=white, ball color=black,smooth] ({\xc},{\yc}) circle ({\rr}) node {3};
%построение 4 шарика 
\draw [rounded corners=14pt,color=white, ball color=black,smooth] ({\xd},{\yd}) circle ({\rr}) node {4};
%построение 5 шарика 
\draw [rounded corners=14pt,color=white, ball color=black,smooth] ({\xe},{\ye}) circle ({\rr}) node {5};
%построение 6 шарика 
\draw [rounded corners=14pt,color=white, ball color=black,smooth] ({\xf},{\yf}) circle ({\rr}) node {6};

%построение граней

%построение грани 1
\draw[line width=6,black]  ({\xa},{\ya})--({\xb},{\yb});
%построение грани 2
\draw[line width=6,black]  ({\xb},{\yb})--({\xc},{\yc});
%построение грани 3
\draw[line width=6,black]  ({\xa},{\ya})--({\xd},{\yd});
%построение грани 4
\draw[line width=6,black]  ({\xc},{\yc})--({\xd},{\yd});
%построение грани 7
\draw[line width=6,black]  ({\xa},{\ya})--({\xd},{\yd});
%построение грани 8
\draw[line width=6,black]  ({\xc},{\yc})--({\xf},{\yf});
%построение грани 9
\draw[line width=6,black]  ({\xd},{\yd})--({\xe},{\ye});
%построение грани 10
\draw[line width=6,black]  ({\xe},{\ye})--({\xf},{\yf});
%\end{scope}
\end{tikzpicture}
\clearpage
\newpage




\section{Литература}
\begin{thebibliography}{2}
	\bibitem{Прокшин Н.А.} Конспект лекций по курсу "Информатика".
	\bibitem{Прокшин Н.А.} Конспект лекций.Практические работы по дисциплине «Информатика»/Сост.: А.Н.Прокшин
СПб.: Изд-во СПбГЭТУ «ЛЭТИ», 2018, 23 с.
\end{thebibliography}
\end{document}
\end{document}